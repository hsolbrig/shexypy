\documentclass{article}
\usepackage{enumitem}
\usepackage{zed}
\usepackage{verbatim}
\usepackage{framed}
\usepackage{fancyvrb}

\def\spec#1{{\tt \small \textsc{{#1}} }}
\def\req#1{{\sf {{\it #1}} }}
\def\bnf#1{{\scriptsize {{#1}} }}
\def\desc#1{{\small \textsl{{#1}} }}

\def\uri#1{{\tt #1}}
\def\triple#1 {\textbf{\textit{ #1}}}

\raggedbottom
% c - concept operators

%%inop \ccup 4
%%inop \ccap 3
\def\ccup{\mathrel{\sqcup}}
\def\ccap{\mathrel{\sqcap}}
\def\cnot{\mathop{\neg}}
\def\csub{\mathrel{\sqsubseteq}}
\def\cequiv{\mathrel{\equiv}}


\def\:{:}

%%inop \csub 1
%%inop \cequiv 1

% r - relation operators

\def\rsub{\mathrel{\sqsubseteq}}
\def\requiv{\mathrel{\equiv}}
%%inop \rsub \atleastone[1
%%inop]uiv 1

%%ignore \ldots 

% quantifiers 

\makeatletter % here come some @s in macro names

\def\newquant#1#2{%
  \@namedef{#1}{#2\@ifnextchar~{\@quantapp}{\relax}}}
\def\@quantapp~#1~#2{#1 \mathbin{.} #2}

\newquant{cforall}{\forall}
\newquant{cexists}{\exists}
\newquant{cge}{\le}
\newquant{cle}{\ge}


% multiplicities for attribute declarations

\def\newmult#1#2{%
  \@namedef{#1}{\@ifnextchar[{\@multapp{#2}}{\relax}}}
\def\@multapp#1[#2]{#2 [#1}

\def\ordlit{\{ordered\}}

\newmult{optional}{0\upto 1\rbrack}
\newmult{many}{0\upto *\rbrack}
\newmult{omany}{0\upto *\rbrack\ordlit}
\newmult{one}{1\ \rbrack}
\newmult{done}{0\upto 1\rbrack}
\newmult{atleastone}{1\upto *\rbrack}
\newmult{oatleastone}{1\upto *\rbrack\ordlit}

\makeatother % done with the @s 

% set of all concepts, set of atomic concepts, set of complex concepts

\def\concept{\Phi}
\def\atomic{\Phi_A}
\def\complex{\Phi_C}

\def\axiom{\mathsf{axiom}}
\def\role{\mathsf{role}}
\def\assertion{\mathsf{assertion}}

\def\TBox{\mathsf{TBox}}
\def\RBox{\mathsf{RBox}}
\def\ABox{\mathsf{ABox}}

%%postop \rinv 
\def\rinv{^{{-}}}

\def\individual{\Upsilon}
\def\assertion{\mathsf{assertion}}

% and more concept operators 

%%inop \cin \cnin \rin \rnin \ieq \ineq 6
\def\cin{\in}
\def\cnin{\mathrel{:\!\neg}}
\def\rin{\in}
\def\rnin{\mathrel{:\!\neg}}
\def\ieq{=}
\def\ineq{\neq}

\def\ltd{<_{DT}}
\def\opt{OPT}

\def\leqlevel{\leq_{level}}

\def\real{\mathsf{\mathbb{R}}}

\def\optopt[#1]{\optional[#1]_{OPT}}
\def\opt[#1]{#1_{OPT}}
\def\optmany[#1]{\many[#1]_{OPT}}
\def\optatleastone[#1]{\atleastone[#1]_{OPT}}

\def\nameIn{\mathrel{nameIn}}
\def\refIn{\mathrel{refIn}}
\def\overwrite{\oplus}
\def\mapIn{\mathrel{mapIn}}
\def\equivTo{\mathrel{==}}




\title{A Declarative Semantics for the ShEx grammar}

\begin{document}
\maketitle
\tableofcontents
\pagebreak


\section{The RDF Data Model}
We begin by defining a formal model of the RDF triple and corresponding graph. Much of the work in this section
derives directly or indirectly from the author's previous work on
\textit{A Formal Model for RDF Dataset Constraints}\cite{jist2013},
\textit{Validating RDF with Shape Expressions}\cite{DBLP:journals/corr/BonevaGHPSS14} as well as
from various submissions and contributions to the 2013 RDF Validation
Workshop\cite{RDFVal} where similar models were used to develop  requirements for a language to specify pre and post conditions for RDF dataset updates as well as to
describe a basic set of predicates which would allow invariants to be asserted about the state of RDF Graphs.

\begin{zed}
[String]
\end{zed}

\subsection{Basic RDF Data Structures}
An \textbf{RDF graph} is a set of RDF Triples:
\begin{zed}
  Graph == \power Triple
\end{zed}

\noindent
An \textbf{RDF triple} consists of three components:
  \begin{itemize}
  \item the subject, which is an IRI or a blank node
  \item the predicate, which is an IRI
  \item the object, which is an IRI, a literal or a blank node
  \end{itemize}

\begin{schema}{Triple}
  s,p,o : RDFTerm
  \where
  iri \inv s \in IRI \lor bnode \inv s \in BNODE \\
   iri \inv p \in IRI
 \end{schema}
 
\noindent
IRIs, literals and blank nodes are collectively known as \textbf{RDF terms}.\\
IRIs, literals and blank nodes are distinct and distinguishable.

\begin{zed}
[IRI, BNODE] \\
RDFTerm ::= iri \ldata IRI \rdata | literal \ldata RDFLiteral \rdata | bnode \ldata BNODE \rdata \\
\end{zed}

\noindent
A \textbf{literal} in an RDF graph consists of two or three elements:
\begin{itemize}
\item a \textbf{lexical form}, ...
\item a \textbf{datatype IRI}, ...
\item if there is a language tag the datatype IRI must be \\
\uri{http://www.w3.org/1999/02/22-rdf-syntax-ns\#langString}
\end{itemize}

\begin{zed}
[LANGTAG] 
\end{zed}
\begin{axdef}
rdf\_langString : IRI
\end{axdef}

\begin{schema}{RDFLiteral}
   v: String \\
   dataType: \optional[IRI] \\
   langTag: \optional[LANGTAG]
\where
  \#langTag > 0 \iff (\#dataType > 0 \land head~dataType = rdf\_langString)
\end{schema}


\subsection{Well Known RDF IRIs}

\begin{axdef}
  RDFGraph : Graph \\
  rdf\_type, rdfs\_subClassOf, rdfs\_subPropertyOf, rdf\_first, rdf\_rest, rdf\_Nil : IRI \\
  rdf\_List, rdf\_Seq, rdf\_Bag, xsd\_string, xsd\_integer, xsd\_double, xsd\_boolean : IRI \\
  rdf\_Alt : IRI 
\end{axdef}

\noindent
Define a set of synonyms to represent the above IRIs as their corresponding RDF terms.
\begin{zed}
  rdf\_type\_t == iri~rdf\_type \\
  rdfs\_subClassOf\_t == iri~rdfs\_subClassOf \\
  rdfs\_subPropertyOf\_t == iri~rdfs\_subPropertyOf \\
  rdf\_first\_t == iri~rdf\_first \\
  rdf\_rest\_t == iri~rdf\_rest \\
  rdf\_Nil\_t == iri~rdf\_Nil \\
  rdf\_Seq\_t == iri~rdf\_Seq \\
  rdf\_List\_t == iri~rdf\_List \\
  rdf\_Bag\_t == iri~rdf\_Bag \\
  rdf\_Alt\_t == iri~rdf\_Alt \\
  xsd\_string\_t == iri~xsd\_string \\
  xsd\_integer\_t == iri~xsd\_integer \\
  xsd\_double\_t == iri~xsd\_double \\
  xsd\_boolean\_t == iri~xsd\_boolean
\end{zed}

\section{Graph Operations}
This section defines a set of ``low level'' functions providing a foundation for making assertions about the members of an RDF Graph. 

\paragraph{\textbf{triplesFor}} - given a graph and an RDF term, return the set of triples where the term appears as a subject
\begin{gendef}
  triplesFor : Graph \pfun RDFTerm \pfun \power Triple
\where
  \forall g:Graph; s : RDFTerm @ triplesFor~g~s = \{t : g | t.s = s \}
\end{gendef}

\paragraph{\textbf{objectsOf}} - given a graph, a subject RDF term and a predicate URI, return the set of objects occurring in triples with the given subject and predicate
\begin{gendef}
   objectsOf : Graph \pfun RDFTerm \pfun RDFTerm \pfun \power RDFTerm
\where
   \forall g : Graph; s, p : RDFTerm @ objectsOf~g~s~p = \{t : triplesFor~g~s | t.p = p @ t.o\}
\end{gendef}

\paragraph{\textbf{objectOf} }- return the unique target of a subject and predicate in a graph or rdf:Nil if there are no or more than one targets
\begin{gendef}
  objectOf : Graph \pfun RDFTerm \pfun RDFTerm \pfun RDFTerm
\where
  \forall g : Graph; s,p: RDFTerm @ objectOf~g~s~p = \\
\t1    \IF \# (objectsOf~g~s~p) \neq 1 \THEN rdf\_Nil\_t \ELSE \\
\t1   (\mu o : RDFTerm | o \in objectsOf~g~s~p)
\end{gendef}

\paragraph{\textbf{collection}}- the set of predicates that are collections of the given subject in the context of the graph.  A predicate represents a collection if
it (a) its object is the subject with a \uri{rdf:type} of \uri{rdf:List}, \uri{rdf:Bag}, \uri{rdf:Seq} or \uri{rdf:Alt} or
(b) it has an \uri{rdf:first} or \uri{rdf:rest} predicate.  Note this is still an approximation of a list because the definition of lists in RDF is surprisingly lax: `` RDFS does not require there be only one first element of a list-like structure, or even for a list-like structure to have a first element. ''\cite{rdfschema}
 
\begin{gendef}
    collection : Graph \pfun RDFTerm \pfun RDFTerm 
\where
    \forall g: Graph; s,p : RDFTerm @ collection~g~s = p \iff \\
\t1 \# (objectsOf~g~s~p) = 1 \land \\
\t1 (\exists o: objectsOf~g~(objectOf~g~s~p)~rdf\_type\_t @ \\
\t2 o \in \{rdf\_List\_t, rdf\_Bag\_t, rdf\_Seq\_t, rdf\_Alt\_t\} ) \lor 
\also
\t1 (\exists p : triplesFor~g~(objectOf~g~s~p) @ \\
\t2 p.o \in \{rdf\_first\_t, rdf\_rest\_t\})
\end{gendef}
 

\paragraph{\textbf{toSeq}} - return the sequence representing whose subject is the supplied RDF term.  If the subject is not declared to be a list type
(\uri{rdf:List}, \uri{rdf:Seq}, \uri{rdf:Bag} or \uri{rdf:Alt} return 
the set of all objects. Otherwise, unwind the list, ignoring extraneous content as ``poorly structured''\footnote{Strict validation should probably declare the entire subject as
invalid, but our goal is to validate UML, not the internal structure of RDF.  Besides, anyone \emph{can} say anything anywhere, right?}


In order to return a sequence of the direct targets of the predicate or more than one \uri{rdf:first} and/or \uri{rdf:next} assertion we define a function that takes a set of RDF Terms
 and returns them as a sequence.  Order is not important, which means we cannot compare the results of any expression utilizing this function(!).
\begin{gendef}
   asSeq : \power RDFTerm \fun \seq RDFTerm \\
\end{gendef}

Unwind combines a sequence constructed of the targets of all of the \uri{rdf:first} predicates not \uri{rdf:Nil} with the sequence resulting
from unwinding all of the non-\uri{rdf:Nil} \uri{rdf:rest} targets.
\begin{gendef}
    unwind : Graph \pfun RDFTerm \pfun \seq RDFTerm \\
    rest : Graph \pfun \seq RDFTerm \pfun \seq RDFTerm
\where
    \forall g : Graph; s : RDFTerm  @ unwind~g~s = \\
\t1 asSeq(\{f:objectsOf~g~s~rdf\_first\_t | f \neq rdf\_Nil\_t\}) \cat \\
\t2 rest~g~(asSeq~\{rst : objectsOf~g~s~rdf\_rest\_t \})
\also
   \forall g: Graph; s : \seq RDFTerm @ rest~g~s = \\
\t1 \IF s = \langle \rangle \THEN \langle  \rangle \\
\t1 \ELSE unwind~g~(head~s) \cat rest~g~(tail~s) 
\end{gendef}

The actual $toSeq$ function:
\begin{gendef}
   toSeq : Graph \pfun RDFTerm \pfun RDFTerm \pfun \seq RDFTerm
\where
   \forall g: Graph; s,p : RDFTerm @ toSeq~g~s~p = \IF collection~g~s = p \\
    \THEN unwind~g~s
    \ELSE asSeq(objectsOf~g~s~p)
\end{gendef}
 
\paragraph{\textbf{isOrdered}} - determine whether a property is an ordered list.

This test reflects some assumptions about RDF possibly deserving
further examination -- the presence of the RDF List Collection
type (see: 5.1 Container Classes and Properties in RDF
Schema\cite{rdfschema})\footnote{Also note this section is
  \emph{not} normative - an indication the community may be
  moving away from the LISP like representation of RDF collections} is
\emph{the} way ordering is represented.  We also make some
assumptions about lists that, while valid in practice, aren't
guaranteed.  One example is we assume a graph of the form
below can never occur.
\begin{Verbatim}[fontsize=\scriptsize, frame=single, fontseries=b]
   @prefix s: <http://sample.org/sample/> .
   @prefix rdf: <http://www.w3.org/1999/02/22-rdf-syntax-ns#> .
   
   s:subj a s:Narwahl;
          foaf:firstName "Jim";
          rdf_first s:n1;
          rdf_rest rdf\_Nil.
\end{Verbatim}
or, more formally that:

\begin{gendef}
   isOrdered : Graph \pfun RDFTerm \pfun RDFTerm
\where
   \forall g: Graph; s, p: RDFTerm @ isOrdered~g~s = p \iff \\
\t1 \# (objectsOf~g~s~rdf\_first\_t) > 0 \land \\
\t1 (\# (objectsOf~g~s~rdf\_rest\_t) > 0 \land \\
\t2 (\# (triplesFor~g~s) = 2) \lor (\# (triplesFor~g~s) = 3 \land \\
\t2 objectOf~g~s~rdf\_type\_t = rdf\_Seq\_t))
\end{gendef}

This isn't strictly true because it states ``RDFS does not require
that there be only one first element of a list-like structure, or even
that a list-like structure have a first element.''\cite{rdfschema}
This may be a place where it would be useful to create some RDF
``meta-schema'' profiles to reduce the amount of optionality.

\section{Interpretation of ShEx Schema}
\subsection{Schema}
\verb|<xs:element name="Schema" type="shex:Schema"/>| \\
\verb|<xs:complexType name="Schema">|

\begin{zed}
ShapeLabel == IRI
\end{zed}

\begin{schema}{Schema}
   startActions : \optional[SemanticActions] \\
   shape : ShapeLabel \pfun Shape \\
   start : \optional[ShapeLabel]
\where
   \ran start \subseteq \dom shape \\
   \# start > 0 \implies (shape~(head~start)).virtual = implementable
\end{schema}

\begin{schema}{Result}
   passedTriples : Graph \\
   excessTriples : Graph
\end{schema}{pass}


\begin{zed}
EvalResult ::= pass \ldata Result \rdata | fail
\end{zed}

\begin{axdef}
   i\_Schema : \optional[ShapeLabel] \fun Schema \fun IRI \fun Graph \fun EvalResult 
\where
   \forall osl : \optional[ShapeLabel]; sch: Schema; iri : IRI; g : Graph  @ \\
   i\_Schema~osl~sch~iri~g = 
\end{axdef}

\begin{schema}{Shape}
   ShapeConstraintGroup \\
   include : \power ShapeLabel \\
   extra : \power IRI \\
   is\_virtual : \optional[Virtual] \\
   is\_closed : \optional[Closed] 
\also
   virtual : Virtual \\
   closed : Closed
\where
   virtual = \IF \# is\_virtual > 0 \THEN head~is\_virtual \ELSE implementable \\
   closed = \IF \# is\_closed > 0 \THEN head~is\_closed \ELSE open
\end{schema}


\begin{zed}
ShapeConstraintGroupChoice ::= \\
\t1 scg\_tripleConstraint \ldata TripleConstraint \rdata | \\
\t1 scg\_oneOf \ldata ShapeConstraint' \rdata | \\
\t1 scg\_someOf \ldata ShapeConstraint' \rdata | \\
\t1 scg\_group \ldata ShapeConstraint' \rdata | \\
\t1 scg\_include \ldata ShapeLabel \rdata
\end{zed}
\begin{zed}
Virtual ::= abstract | implementable \\
Closed ::= fixed | open \\
\end{zed}

\begin{schema}{ShapeConstraintGroup}
   entry : \power ShapeConstraintGroupChoice \\
   semanticActions :  \optional[SemanticActions] 
\end{schema}
\subsection{TripleConstraint}

\begin{schema}{TripleConstraint}
   Cardinality \\
   subjectConstraint : \optional[ValueClass] \\
   predicate : IRI \\
   objectConstraint : \optional[ValueClass] \\
   annotation : \power Annotation \\
   semanticActions : \optional[SemanticActions] \\
   label : \optional[ShapeLabel] \\
\where
   \# subjectConstraint + \# objectConstraint \leq 1
\end{schema}

$NodeType$ comes from the triple model

\begin{zed}
AnnotationChoice ::= \\
\t1 annot\_iriref \ldata IRI \rdata | \\
\t1 annot\_literal \ldata RDFLiteral \rdata
\end{zed}

\begin{schema}{Annotation}
	iri : IRI \\
	annot : AnnotationChoice
\end{schema}




\subsection{Cardinality}
\verb|<xs:attributeGroup name="Cardinality">|
\begin{zed}
UnboundedInt ::= num \ldata \nat \rdata | unlimited
\end{zed}
\begin{schema}{Cardinality}
   min\_v : \optional[\nat] \\
   max\_v : \optional[UnboundedInt] \\
   min : \nat \\
   max : UnboundedInt
\where
   min = \IF \# min\_v > 0 \THEN head~ min\_v \ELSE 1 \\
   max = \IF \# max\_v > 0 \THEN head~ max\_v \ELSE num~ 1
\end{schema}
\begin{zed}
[ShapeConstraint']
\end{zed}


\begin{schema}{ShapeConstraint}
   ShapeConstraintGroup \\
   Cardinality \\
   label : \optional[ShapeLabel]
\end{schema}




\begin{schema}{GroupShapeConstr}
	ref : \power ShapeLabel \\
	stringFacet : \power StringFacet
\end{schema}

\begin{zed}
endType ::= exclusive | inclusive
\end{zed}

\begin{schema}{Range}
   v: \nat \\
   is\_open : \optional[endType] \\
   open : endType
\where
   open = \IF \# is\_open > 0 \THEN head~ is\_open \ELSE inclusive
\end{schema}
  
\subsection{Facets}
\begin{zed}
StringFacet ::=  \\
\t2 sf\_pattern \ldata String \rdata | \\
\t2 sf\_not \ldata String \rdata  | \\
\t2 sf\_minLength \ldata \nat \rdata  | \\
\t2 sf\_maxLength \ldata \nat \rdata |  \\
\t2 sf\_length \ldata \nat \rdata  
\end{zed}

\begin{zed}
NumericFacet ::= \\
\t2 nf\_minValue \ldata Range \rdata  | \\
\t2 nf\_maxValue \ldata Range \rdata  | \\
\t2 nf\_totalDigits \ldata \nat \rdata |  \\
\t2 nf\_fractionDigits \ldata \nat \rdata
\end{zed}

\begin{zed}
XSFacet ::=  \\
\t2 xsf\_pattern \ldata String \rdata | \\
\t2 xsf\_not \ldata String \rdata  | \\
\t2 xsf\_minLength \ldata \nat \rdata  | \\
\t2 xsf\_maxLength \ldata \nat \rdata |  \\
\t2 xsf\_length \ldata \nat \rdata  | \\
\t2 xsf\_minValue \ldata Range \rdata  | \\
\t2 xsf\_maxValue \ldata Range \rdata  | \\
\t2 xsf\_totalDigits \ldata \nat \rdata |  \\
\t2 xsf\_fractionDigits \ldata \nat \rdata
\end{zed}

\begin{zed}
ValueClassChoice ::= \\
\t2 facet \ldata XSFacet \rdata | \\
\t2 groupShapeConstr \ldata GroupShapeConstr \rdata | \\
\t2 valueSet \ldata ValueSet \rdata
\end{zed}

\subsection{ValueClass}
\begin{schema}{ValueClass}
	entry : \power_1 ValueClassChoice
\end{schema}

\subsection{ValueSet}
\begin{schema}{ValueSet}
   filler: \nat
\end{schema}

\subsection{SemanticActions}
\begin{zed}
[CODE\_LABEL, CODE\_BODY]  \\
CODE\_DECL \defs [iri: \optional[IRI]; body: CODE\_BODY] \\
SemanticAction ::= codeLabel \ldata CODE\_LABEL \rdata | \\
\t1 codeDecl \ldata CODE\_DECL \rdata \\
SemanticActions == \power SemanticAction
\end{zed}




\section{Optional elements}
Representing optional elements of type $T$.  Representing it as a sequence allows us to
determine absence by $\#T = 0$ and the value by $head~T$. 

\begin{zed} 
  \optional[T] == \{ s : \seq T \mid \# s \leq 1 \} 
\end{zed}



\end{document}